% !TeX TS-program = lualatex

\documentclass[tikz,border=10pt]{standalone}

\usepackage{fontspec}  % allows for system font usage
\setmainfont{Arial}    % Sets the main font to Arial

\usepackage{xcolor}
\usetikzlibrary{calc}
\usetikzlibrary{positioning}
\usetikzlibrary{arrows.meta}
\usetikzlibrary{decorations.pathreplacing}
\usetikzlibrary{matrix}
\usetikzlibrary{graphs}
\usetikzlibrary{graphs.standard}
\usetikzlibrary{quotes}
\usetikzlibrary{fit}
\usetikzlibrary{chains}
\usetikzlibrary{shapes.geometric}
\usetikzlibrary{shapes.multipart}
\usetikzlibrary {backgrounds}

\usepackage{listofitems} % for \readlist to create arrays

\pgfdeclarelayer{middle}
\pgfsetlayers{background,middle, main}

\begin{document}

\definecolor{c1}{RGB}{239,243,255}
\definecolor{c2}{RGB}{198,219,239}
\definecolor{c3}{RGB}{158,202,225}
\definecolor{c4}{RGB}{107,174,214}
\definecolor{c5}{RGB}{66,146,198}
\definecolor{c6}{RGB}{33,113,181}
\definecolor{c7}{RGB}{8,69,148}


\definecolor{p1}{RGB}{254,224,210}
\definecolor{p2}{RGB}{252,146,114}
\definecolor{p3}{RGB}{222,45,38}

\definecolor{l1}{RGB}{117,107,177}

\colorlet{myred}{red!80!black}
\colorlet{myblue}{blue!80!black}
\colorlet{mygreen}{green!60!black}


\begin{tikzpicture}[
	font=\scriptsize \ttfamily,
	block/.style={rectangle, text centered, rounded corners, minimum height=1em, minimum width=4em, fill=c6},
	connect/.style={rounded corners, black!30, text=black, thick, -{Stealth[length=5pt, width=4pt]}, shorten >=1pt, shorten <=1pt},
	skip loop/.style  2 args= {connect, to path={-- ++(#1, 0) |- node[right, sloped, pos=0.2, inner sep=1pt, fill=white] {#2} (\tikztotarget)}},
	nnode/.style={thin, circle,draw=c7, minimum size=6, inner sep=0.5,outer sep=0.6},
	node in/.style={nnode, green!20!black,draw=c7!30!black,fill=c7!25},
	node hidden/.style={
			nnode,
			blue!20!black,
			draw=myblue!30!black,
			fill=myblue!20,
		},
	node out/.style={nnode, red!20!black, draw=myred!30!black,fill=myred!20},
	]

	% Exitron and Exon
	\begin{scope}[local bounding box=exon, scale=0.9, transform shape]
		\begin{scope}[start chain, node distance=0mm,
				nodes/.style={inner sep=0, outer sep=0},
			]
			\node (exon-n1) [on chain] {ATT};
			\node (exon-n2) [fill=c7, text=white, on chain] {GT$\cdots$AG};
			\node (exon-n3) [on chain] {TCGG};
			\node (exon-n4) [fit=(exon-n1)(exon-n2)(exon-n3), draw=c7,  thick, inner sep=0] {};
			\node (exitron-label) [above=1pt of exon-n4] {Exitron};
			\node (exitron-label2) [above right= of exon-n4, xshift=-7] {Exon};
		\end{scope}

		\node (left) [left= 20pt of exon-n1]  {};
		\node (right) [right= 11pt of exon-n3]   {};

		\path (left) edge[-, thick]  (exon-n1)
		(exon-n3) edge[-, thick] (right);
		\node (dnalabel) [above=8pt of exon-n1, xshift=-12pt] {Ultra-long DNA};
	\end{scope}


	% RNAseq Matrix
	\begin{scope}[shift={(exon.center)},
			yshift=-1.2cm, xshift=8,
			node distance=0mm, scale=0.7, transform shape]
		\matrix (rnaseq) [matrix of nodes, nodes in empty cells,
			column sep=-\pgflinewidth, row sep=1pt,
			nodes={thin, inner sep=0pt, outer sep=0pt, minimum width=0.6em, minimum height=0.6em, anchor=center},
			row 1/.style={nodes={fill=p1}},
			row 1 column 4/.style={nodes={fill=p3}},
			row 1 column 5/.style={nodes={fill=p3}},
			row 1 column 6/.style={nodes={fill=p3}},
			row 1 column 7/.style={nodes={fill=p3}},
			row 1 column 8/.style={nodes={fill=p3}},
		] {
			0 & 0 & 0 & 1 & 1 & $\cdots$ & 1 & 1 & 0 & 0 & 0 & 0 \\
			A & T & T & G & T & $\cdots$ & A & G & T & C & G & G \\
		};
		\node (rnaseq-label) [left=1pt of rnaseq-2-1] {DNA};
		\node (label) [left=1pt of rnaseq-1-1] {Label};
		\node (databox) [fit=(rnaseq)(rnaseq-label)(label), draw=c7, dashed, thick, inner sep=0] {};
		% \node (rna-input-label) [above=1pt of databox] {\tiny \ttfamily Input};
	\end{scope}


	\begin{scope}[local bounding box=tokens, shift={(databox.south)}, yshift=-0.65cm]
		\matrix (tokens1)   [
			matrix of nodes,
			nodes in empty cells,
			nodes={fill=c2, draw=black, thin, outer sep=0pt},
			column sep=-\pgflinewidth, row sep=-\pgflinewidth,
			column 1/.style={nodes={fill=c1}},
			column 2/.style={nodes={fill=c2}},
			column 3/.style={nodes={fill=c3}},
			column 4/.style={nodes={fill=c4}},
			column 5/.style={nodes={fill=c5}},
			column 6/.style={nodes={fill=c6}},
			column 7/.style={nodes={fill=c7}},
		]
		{
			 &  &  &  &  &  & \\
		};
	\end{scope}

	% \node (lcglm) [below of=tokens1, yshift=0.2cm, block, fill=l1, opacity=1,  thick, minimum height = 3em] {Long-Context GLM};


	\begin{scope}[shift={(tokens1)}, yshift=-1.2cm, xshift=-1.35cm,
		edge1/.style={black, line width=0.6pt, -{Stealth[length=3pt, width=2pt]}, shorten >=0.5pt, shorten <=0.5pt},
		local bounding box=lcglm, scale=0.44, transform shape]
		\graph [
			grow right = 1cm,
			edges = {edge1},
			skip loop/.style={to path={ -- ++(1cm, 0) -| (\tikztotarget) \tikztonodes}},
			hyena/.style ={rectangle, draw=c7, fill=c7!20, thick,  minimum height=5em, minimum width=2em},
			residual/.style={circle, draw=c7, line width=0.6pt, inner sep=1, as=+},
			block1/.style={rectangle, draw=c5, fill=c5!70, thick,   minimum height=3em, minimum width=1em, as=},
			block2/.style={rectangle, draw=c6, fill=c6!70, thick,  minimum height=4em, minimum width=1em, as=},
		] {
			input [as=] -> hyena [hyena] -> r1 [residual] -> b1 [block1] -> b2 [block2] -> r2 [residual] -> b3 [block1];
		};

		\draw[edge1]  ($(input) !.2! (hyena)$) |- ++(0,1) -| (r1);
		\draw[edge1] ($(b1) !.4! (b2) $) |- ++(0,1) -| (r2);
	\end{scope}


	\begin{pgfonlayer}{middle}
		\node (lcglmbox) [fit=(lcglm), fill=l1,  thick, inner ysep=6pt, inner xsep=2pt, yshift=5pt, opacity=0.7] {};
		\node[inner sep=3pt, anchor=north west] at (lcglmbox.north west) {Long-Context GLM};
	\end{pgfonlayer}


	\begin{scope}[shift={(lcglmbox.south)}, yshift=-0.5cm, xshift=3, local bounding box=features]
		\begin{scope}[opacity=0.8]
			\matrix (features1)   [
				matrix of nodes,
				nodes in empty cells,
				nodes={fill=c2,draw},
				column sep=-\pgflinewidth, row sep=-\pgflinewidth,
				column 1/.style={nodes={fill=c1}},
				column 2/.style={nodes={fill=c2}},
				column 3/.style={nodes={fill=c3}},
				column 4/.style={nodes={fill=c4}},
				column 5/.style={nodes={fill=c5}},
				column 6/.style={nodes={fill=c6}},
				column 7/.style={nodes={fill=c7}},
			]
			{
				 &  &  &  &  &  & \\
			};
		\end{scope}

		\begin{scope}[shift={(-0.1, -0.1)}, opacity=0.9]
			\matrix (features2)   [
				matrix of nodes,
				nodes in empty cells,
				nodes={fill=c2, draw, outer sep=0pt},
				column sep=-\pgflinewidth, row sep=-\pgflinewidth,
				column 1/.style={nodes={fill=c1}},
				column 2/.style={nodes={fill=c2}},
				column 3/.style={nodes={fill=c3}},
				column 4/.style={nodes={fill=c4}},
				column 5/.style={nodes={fill=c5}},
				column 6/.style={nodes={fill=c6}},
				column 7/.style={nodes={fill=c7}},
			]
			{
				 &  &  &  &  &  & \\
			};
		\end{scope}

		\begin{scope}[shift={(-0.2, -0.2)}, opacity=1]
			\matrix (features3)   [
				matrix of nodes,
				nodes in empty cells,
				nodes={fill=c2, draw, outer sep=0pt},
				column sep=-\pgflinewidth, row sep=-\pgflinewidth,
				column 1/.style={nodes={fill=c1}},
				column 2/.style={nodes={fill=c2}},
				column 3/.style={nodes={fill=c3}},
				column 4/.style={nodes={fill=c4}},
				column 5/.style={nodes={fill=c5}},
				column 6/.style={nodes={fill=c6}},
				column 7/.style={nodes={fill=c7}},
			]
			{
				 &  &  &  &  &  & \\
			};
		\end{scope}

	\end{scope}

	\begin{scope}[shift={(lcglm)}, yshift=-1.2cm, x=2.2cm, y=1.4cm, local bounding box=mlp,
			mynode/.style={thick, draw=c7,fill=c7!20, circle, minimum size=18},
			rotate=-90,
			scale=0.3,
			transform shape,
		]
		\readlist\Nnod{3,4,2} % number of nodes per layer
		% \Nnodlen = length of \Nnod (i.e. total number of layers)
		% \Nnod[1] = element (number of nodes) at index 1
		\foreachitem \N \in \Nnod{ % loop over layers
		% \N     = current element in this iteration (i.e. number of nodes for this layer)
		% \Ncnt  = index of current layer in this iteration
		\foreach \i [evaluate={\x=\Ncnt; \y=\N/2-\i+0.5; \prev=int(\Ncnt-1);}] in {1,...,\N}{ % loop over nodes
		\node[mynode] (N\Ncnt-\i) at (\x,\y) {};
		\ifnum\Ncnt>1 % connect to previous layer
			\foreach \j in {1,...,\Nnod[\prev]}{ % loop over nodes in previous layer
			\draw[draw=black!80,  line width=0.6pt, -{Stealth[length=3pt, width=2pt]},
			shorten >=0.5pt, shorten <=0.5pt] (N\prev-\j) -- (N\Ncnt-\i); % connect arrows directly
			}
		\fi % else: nothing to connect first layer
		}
		}
	\end{scope}

	\node [left = of N2-1, xshift=-10pt, yshift=-2pt] {MLP};


	\begin{scope}[local bounding box=outputbox, shift={(mlp.south)}, yshift=-0.6cm]
		\matrix (output)   [
			matrix of nodes,
			nodes in empty cells,
			nodes={fill=c2, draw=black, thin, outer sep=0pt},
			column sep=-\pgflinewidth, row sep=-\pgflinewidth,
			column 1/.style={nodes={fill=p1}},
			column 2/.style={nodes={fill=p1}},
			column 3/.style={nodes={fill=p3}},
			column 4/.style={nodes={fill=p3}},
			column 5/.style={nodes={fill=p1}},
			column 6/.style={nodes={fill=p1}},
			column 7/.style={nodes={fill=p1}},
		]
		{
			 &  &  &  &  &  & \\
		};
	\end{scope}

	\node [left = of output, xshift=0.65cm, anchor=center] {Output};

	\begin{scope}[shift={(output)}, yshift=-0.8cm, local bounding box=prediction, scale=1, transform shape]
		\matrix (mprediction) [matrix of nodes, nodes in empty cells,
			column sep=-\pgflinewidth, row sep=-\pgflinewidth,
			nodes={thin,  inner sep=0pt, outer sep=0pt, minimum width=0.6em, minimum height=0.6em, anchor=center},
			row 1 column 4/.style={nodes={fill=p3, text=white}},
			row 1 column 5/.style={nodes={fill=p3, text=white}},
			row 1 column 6/.style={nodes={fill=p3, text=white}},
			row 1 column 7/.style={nodes={fill=p3, text=white}},
			row 1 column 8/.style={nodes={fill=p3, text=white}},
		] {
			A & T & T & G & T & $\cdots$ & A & G & T & C & G & G \\
		};
	\end{scope}


	\path
	(exon) edge[connect] (databox)
	(databox) edge["Tokenizer", connect] (tokens)
	(mlp) edge[connect] (output)
	(features) edge[connect] (mlp)
	% (lcglmbox) edge[connect] (features)
	(output) edge["Predict", connect] (prediction)
	edge[blue,  skip loop={18mm}{Loss}] (rnaseq-1-12);

	\draw [connect] (tokens1.south) |- ++(-4.5em, -0.25em) |-  ($(input) + (0.1em, 0)$);

	\draw [connect] ($(b3.east) + (1pt, 0)$) -| ++ (0.5em, -1.3em)  -|   ($(features1) + (-0.3em, 0.3em)$);


	\begin{scope}[on background layer]
		\node (modelbox) [fit=(tokens)(lcglm)(features)(mlp), fill=c2,  thick, inner sep=8pt] {};
		\node[inner sep=2pt, anchor=north west] at (modelbox.north west) {Model};
	\end{scope}

\end{tikzpicture}

\end{document}
