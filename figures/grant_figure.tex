\documentclass[tikz,border=10pt]{standalone}

\usepackage{xcolor}
\usetikzlibrary{calc}
\usetikzlibrary{positioning}
\usetikzlibrary{arrows.meta}
\usetikzlibrary{decorations.pathreplacing}
\usetikzlibrary{matrix}
\usetikzlibrary{graphs}
\usetikzlibrary{graphs.standard}
\usetikzlibrary{quotes}
\usepackage{listofitems} % for \readlist to create arrays
\usepackage{amssymb}
\usepackage{textcomp}  % For \texttimes
\usepackage{relsize}
\begin{document}

\definecolor{c1}{RGB}{239,243,255}
\definecolor{c2}{RGB}{198,219,239}
\definecolor{c3}{RGB}{158,202,225}
\definecolor{c4}{RGB}{107,174,214}
\definecolor{c5}{RGB}{66,146,198}
\definecolor{c6}{RGB}{33,113,181}
\definecolor{c7}{RGB}{8,69,148}


\definecolor{p1}{RGB}{254,224,210}
\definecolor{p2}{RGB}{252,146,114}
\definecolor{p3}{RGB}{222,45,38}

\definecolor{l1}{RGB}{117,107,177}

\colorlet{myred}{red!80!black}
\colorlet{myblue}{blue!80!black}
\colorlet{mygreen}{green!60!black}


\begin{tikzpicture}[
	font=\scriptsize,
	fontscale/.style = {font=\relsize{#1}},
	block/.style={rectangle, text centered, rounded corners, minimum height=1em, minimum width=4em, fill=c6},
	connect/.style={rounded corners, black!65, text=black, very thick, -Stealth[round], shorten >=1pt, shorten <=1pt},
	skip loop/.style  2 args= {connect, to path={-- ++(#1, 0) |- node[right, sloped, pos=0.2, inner sep=1pt, fill=white] {#2} (\tikztotarget)}},
	nnode/.style={thin, circle,draw=c7, minimum size=6, inner sep=0.5,outer sep=0.6},
	node in/.style={nnode, green!20!black,draw=c7!30!black,fill=c7!25},
	node hidden/.style={
			nnode,
			blue!20!black,
			draw=myblue!30!black,
			fill=myblue!20,
		},
	node out/.style={nnode, red!20!black, draw=myred!30!black,fill=myred!20},
	]

	\node (data) [block] {Data};

	\begin{scope}[local bounding box=tokens, shift={(data.south)}, yshift=-1cm]
		\matrix (tokens1)   [
			matrix of nodes,
			nodes in empty cells,
			nodes={fill=c2, draw=black, thin, outer sep=0pt},
			column sep=-\pgflinewidth, row sep=-\pgflinewidth,
			column 1/.style={nodes={fill=c1}},
			column 2/.style={nodes={fill=c2}},
			column 3/.style={nodes={fill=c3}},
			column 4/.style={nodes={fill=c4}},
			column 5/.style={nodes={fill=c5}},
			column 6/.style={nodes={fill=c6}},
			column 7/.style={nodes={fill=c7}},
		]
		{
			 &  &  &  &  &  & \\
		};
	\end{scope}

	\node (lcglm) [below of=tokens1, yshift=-0.7cm, block, fill=l1,  thick, minimum height = 3em] {Long-Context GLM};

	\begin{scope}[shift={(lcglm.south)}, yshift=-1cm, xshift=3, local bounding box=features]
		\begin{scope}[opacity=0.8]
			\matrix (features1)   [
				matrix of nodes,
				nodes in empty cells,
				nodes={fill=c2,draw},
				column sep=-\pgflinewidth, row sep=-\pgflinewidth,
				column 1/.style={nodes={fill=c1}},
				column 2/.style={nodes={fill=c2}},
				column 3/.style={nodes={fill=c3}},
				column 4/.style={nodes={fill=c4}},
				column 5/.style={nodes={fill=c5}},
				column 6/.style={nodes={fill=c6}},
				column 7/.style={nodes={fill=c7}},
			]
			{
				 &  &  &  &  &  & \\
			};
		\end{scope}

		\begin{scope}[shift={(-0.1, -0.1)}, opacity=0.9]
			\matrix (features2)   [
				matrix of nodes,
				nodes in empty cells,
				nodes={fill=c2, draw, outer sep=0pt},
				column sep=-\pgflinewidth, row sep=-\pgflinewidth,
				column 1/.style={nodes={fill=c1}},
				column 2/.style={nodes={fill=c2}},
				column 3/.style={nodes={fill=c3}},
				column 4/.style={nodes={fill=c4}},
				column 5/.style={nodes={fill=c5}},
				column 6/.style={nodes={fill=c6}},
				column 7/.style={nodes={fill=c7}},
			]
			{
				 &  &  &  &  &  & \\
			};
		\end{scope}

		\begin{scope}[shift={(-0.2, -0.2)}, opacity=1]
			\matrix (features3)   [
				matrix of nodes,
				nodes in empty cells,
				nodes={fill=c2, draw, outer sep=0pt},
				column sep=-\pgflinewidth, row sep=-\pgflinewidth,
				column 1/.style={nodes={fill=c1}},
				column 2/.style={nodes={fill=c2}},
				column 3/.style={nodes={fill=c3}},
				column 4/.style={nodes={fill=c4}},
				column 5/.style={nodes={fill=c5}},
				column 6/.style={nodes={fill=c6}},
				column 7/.style={nodes={fill=c7}},
			]
			{
				 &  &  &  &  &  & \\
			};
		\end{scope}
	\end{scope}


	\node (mlp) [below of=features2, yshift=-0.5cm, rectangle, text centered, rounded corners, minimum width=5em, minimum height=2em,  fill=c6] {MLP};

	\begin{scope}[local bounding box=outputbox, shift={(mlp.south)}, yshift=-1cm]
		\matrix (output)   [
			matrix of nodes,
			nodes in empty cells,
			nodes={fill=c2, draw=black, thin, outer sep=0pt},
			column sep=-\pgflinewidth, row sep=-\pgflinewidth,
			column 1/.style={nodes={fill=p1}},
			column 2/.style={nodes={fill=p1}},
			column 3/.style={nodes={fill=p3}},
			column 4/.style={nodes={fill=p3}},
			column 5/.style={nodes={fill=p3}},
			column 6/.style={nodes={fill=p1}},
			column 7/.style={nodes={fill=p1}},
		]
		{
			 &  &  &  &  &  & \\
		};
	\end{scope}



	% Node styles
	\tikzset{
		yesStyle/.style={
				circle, draw=green!60, fill=green!5, thick,
				minimum size=1, % Define the size of the circle
				inner sep=1pt, % Minimal padding around the text
				text centered % Center the content
			},
		noStyle/.style={
				rectangle, draw=red!60, fill=red!5, thick,
				minimum size=1, % Ensure it matches the yesStyle
				inner sep=2pt, % Consistent with yesStyle
				text centered % Center the content
			}
	}

	\node (prediction) [below=of outputbox] {};  % Text node
	% Yes and No nodes with custom shapes
	\node (prediction1) [yesStyle, below = of output, xshift=-1em , scale=0.8]{\checkmark};  % Using tick symbol
	\node (prediction2) [noStyle, below = of output, xshift=1em, scale=0.8] {\texttimes};  % Using cross symbol



	\path (data) edge["Tokenizer", connect, font=\tiny] (tokens)
	(tokens) edge[connect] (lcglm)
	(lcglm) edge[connect] (features)
	(features) edge[connect] (mlp)
	(mlp) edge[connect] (output)
	% (output) edge[connect] (prediction)
	(output) edge["Predict", connect, font=\tiny] (prediction)
	edge[blue,  skip loop={15mm}{Loss}] (data);
\end{tikzpicture}

\end{document}
